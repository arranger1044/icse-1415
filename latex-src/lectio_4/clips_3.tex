\documentclass[xcolor={usenames,dvipsnames,svgnames}, compress]{beamer}

% \usepackage[utf8]{inputenc}
\usepackage{booktabs}
\usepackage{dcolumn}
\usepackage{colortbl}

% \usepackage[style=authoryear-comp, backref=true]{biblatex}
\usepackage{ifxetex}
\usepackage{amsmath}
\usepackage{biblatex}
% 
\usepackage[no-math]{fontspec}


% 
% CLIPS listings
\input{../clips-listings}

\definecolor{jess-fucsia}{RGB}{170, 0, 127}


\hypersetup{
  colorlinks=true,       % false: boxed links; true: colored links
  linkcolor=jess-fucsia,          % color of internal links (change box color with linkbordercolor)
  % citecolor=green,        % color of links to bibliography
  %filecolor=magenta,      % color of file links
  urlcolor=jess-fucsia           % color of external links
}

%%% Local Variables:
%%% mode: latex
%%% TeX-master: t
%%% End:


\usepackage{lacamlisciotheme/beamerthemelacamliscio}



% \addbibresource{../referomnia/referomnia.bib}

\setbeamertemplate{headline}{}

%%%%%%%%%%%%%%%%%%%%%%%%%%%%%%%%%%%%%%%%%%%%%%%%%%%%%%%%%%%% 
%%%%%%%%%%%%%%%%%%%%%%%%%%%%%%%%%%%%%%%%%%%%%%%%%%%%%%%%%%%% 
%%%%%%%%%%%%%%%%%%%%%%%%%%%%%%%%%%%%%%%%%%%%%%%%%%%%%%%%%%%% 


\begin{document}

% \title{Metodi di Apprendimento Statistico-Relazionale: Sum-Product Network \& Co-clustering}
\title{CLIPS: Knowledge Representation and Problem Solving}
%\subtitle{An Introduction}
\author{Antonio Vergari}
% \institute{Lacam$@$DIB$@$Uniba}
\institute{Università degli Studi di Bari}
\department{Dipartimento di Informatica}
\laboratory{LACAM}
\group{Machine Learning}
\institutelogo{\includegraphics[width=25pt]{Figures/unibaba}}
\lablogo{\includegraphics[width=35pt]{Figures/lacam}}

\footnotesize \let\small\footnotesize





{
  \setbeamertemplate{headline}{}
  \setbeamertemplate{footline}{}
  \begin{frame}
    \titlepage
  \end{frame}
}

\section{Knowledge Representation}
{\setbeamertemplate{headline}{}
  \begin{frame}
    \sectionpage
  \end{frame}
}

\begin{frame}
  \frametitle{A simple diagnostic classifier}
  Getting back to the previous example, the knowledge we wanted to
  embed could be expressed in such a few rules:
  \begin{enumerate}[I.]
  \item \textbf{If} the skin and eyes are yellowish, \textbf{then}
    there is \emph{icterus} (partial diagnosis)
  \item  \textbf{If} the eyes are yellowish but not the skin, \textbf{then} there is
    \emph{scleral icterus} (partial diagnosis)
  \item \textbf{If} scleral icterus is present, there is no fever, and the
    patient's stressed or without food, \textbf{then} \emph{Gilbert's syndrome} can be
    diagnosed
  \item  \textbf{If} icterus is present, there is fever, the patient is young
    and tired, dyspepsia has been diagnosed and the liver is enlarged,
    \textbf{then} \emph{Acute viral hepatitis} can be diagnosed
  \item  \textbf{If} there is icterus, fever and the patient is not young but
    has recurrent pains and cholecyst pains, \textbf{then} it is  \emph{cholecystitis}
  \item \textbf{If} there is icterus, there is no fever and the patient is not
    young and abuses alchool, the liver is enlarged and the spleen is
    enlarged as well, \textbf{then} it is \emph{alchoolic cirrhosis}
  \end{enumerate}

\end{frame}

\begin{frame}[fragile]
  \frametitle{A very simple fact representation}
  The minimal information needed to represent a symptom as a fact is
  its name and its presence. In this way we are limiting possible
  extensions (like user profiling), plus each precondition is treated
  as a symptom.
  \begin{clips-code}[numbers=none]
    (deftemplate symptom
        (slot name (type SYMBOL))
        (slot observed (default FALSE)))
  \end{clips-code}

  Thus a diagnostic rule could take this linear form:
  \begin{clips-code}[numbers=none]
    (defrule alchoolic-cirrhosis
        (symptom (name icterus) (observed TRUE))
        (symptom (name fever) (observed FALSE))
        (symptom (name young) (observed FALSE))
        (symptom (name alcohol-abuse) (observed TRUE))
        (symptom (name enlarged-liver) (observed TRUE))
        (symptom (name enlarged-spleen) (observed TRUE))
        =>
        (assert (diagnosis "Alchoolic cirrhosis")))
  \end{clips-code}
  
  In the case of partial diagnosis we will assert symptom facts and
  not diagnosis (this is a limitation as well).
\end{frame}

\begin{frame}
  \frametitle{Rule Hierarchy}
  To exend this simple formalism one can conceptualize in more depth
  what some symptoms really mean and how to observe them.\par\bigskip

  For instance, what does it mean for a  patient to be young? Or to
  have the fever?\par\bigskip
  \textbf{If} the patient's age is less than 25 years, \textbf{then} he's young, \textbf{otherwise if}
  it is still less than 50, \textbf{then} it is an adult, \textbf{else} an
  elder.\par
  \textbf{If} the patient's temperature is higher than 37.1 Celsius degrees,
  \textbf{then} he has a fever.\par \bigskip

  

  We are clearly defining different stages in the diagnosis process,
  refining available information and collecting missing
  information. At the same time the interaction (via questions) shall
  be more focused.\par
   
\end{frame}

\begin{frame}[fragile]
  \frametitle{Randomizing order}
  Our naive conversation behaviour could be more banal if we could
  start the conversation from any possible rule.\par\bigskip
  We can achieve this by imposing a random conflict resolution
  strategy (all rules to ask questions shall have the same priority):
  \begin{clips-code}
    (clear)
    (set-strategy random)
    (load icterus.clp)
    (reset)
    (run)
  \end{clips-code}

  However, please note that a real diagnostic interview can start from
  every point, \emph{but then it is then guided by
  the partial diagnosed diseases and symptoms}. In order to do this we
have to introduce meta-rules\dots\par\bigskip

  \textbf{If} there is a symptom that can be asked that is characterizing a
  disease we could diagnose, \textbf{then} ask about that symptom
\end{frame}

% \begin{frame}
%   \frametitle{Coping with Uncertainty}
  
% \end{frame}

\begin{frame}[fragile]
  \frametitle{Revising Asserted Knowledge}
  If we assume each asserted symptom to be erroneously observed, we
  could provide a routine that lets you change the truth value for
  them all, calling inference again on the new WM.\par
  \begin{clips-code}
    Would you like to revise the diagnosis? (yes/y/no/n): yes
    Would you like to change some observed symptom?
    (1) recurrent-pain: TRUE
    (2) tired: TRUE
    (3) enlarged-liver: TRUE
    (4) yellowish-skin: TRUE
    ...
    (11) enlarged-spleen: TRUE
    (12) alcohol-abuse: TRUE
    (13) cholecyst-pain: TRUE
    (14) fever: TRUE
    Enter the symptom number or 'e' to return or 'h' to stop:
  \end{clips-code}

  Suppose you are setting yellowish-skin to FALSE, will the scleral
  icterus rule fire? \emph{what if scleral icterus has already been asserted before}? 
\end{frame}

\begin{frame}[fragile]
  \frametitle{Revising Asserted Knowledge}
  A simple, manual approach:
  \begin{clips-code}
    (deffunction get-all-facts-by-names ($?template-names)
        (bind ?facts (create$))
        (progn$ (?f (get-fact-list)) ;; this is a foreach
            (if (member$ (fact-relation ?f) $?template-names)
                then (bind ?facts (create$ ?facts ?f)))) ?facts)
    
    (deffunction change-symptom-by-index (?index)
        (bind ?f (nth ?index (get-all-facts-by-names symptom)))
        (modify ?f (observed (not (fact-slot-value ?f observed)))))

    (deffunction ask-to-change-symptom (?question)
        (printout t "Would you like to change some observed symptom?" crlf)
        (print-all-symptoms-status)
        (bind ?n (length$ (get-all-facts-by-names symptom)))
        (bind ?response (ask-question ?question (create$ (gen-int-list ?n) e h)))
        (switch ?response  (case h then (printout t "Halt" crlf) (halt))
                           (case e then (return)) 
                           (default (change-symptom-by-index ?response)
        (ask-to-change-symptom ?question))))
  \end{clips-code}
\end{frame}

\section{Searching}
{\setbeamertemplate{headline}{}
  \begin{frame}
    \sectionpage
  \end{frame}
}

\begin{frame}
  \frametitle{Search State Representation}
  We can represent the inference problem as the search in the space of
  possible diagnosed states.\par
  To implement a problem solver in CLIPS we have to model, for each
  problem\footnote{How could we model a \emph{general problem solver}?}:
  \begin{description}
  \item[\textbf{states}] that form our search space (e.g. problem configurations)
  \item[\textbf{transitions}] that allow to jump from one state to its
    neighbors, the next states (e.g. rules in a game)
  \item[\textbf{constraints}] eventually imposed on states (e.g. no
    repeated or illegal states) and to check for a state to be a solution
    \item[\textbf{search strategy}] that controls which transition to
      apply and which state to move on at each time
    \item[\textbf{explored tree}] representing the states explored so far
  \end{description}
\end{frame}

\begin{frame}
  \frametitle{A tentative implementation}
  The simplest implementation we can try relies on the direct
  exploitation of the constructs that CLIPS offers, using in a
  non-transparent way its inference cycle\par\bigskip
  To be more precise we would have:
  \begin{description}
  \item[\textbf{states}] as facts in the current configuration of the WM.
  \item[\textbf{transitions}] encoded directly as CLIPS rules with the
    same priority
  \item[\textbf{constraints}] as constrained pattern in the LHS of
    transitions (e.g. the \textsf{not} logical operator). Some rules
    can be explicitly used for solution states
  \item[\textbf{search strategy}] directly relying on CLIPS conflict
    resolution strategy
  \item[\textbf{explored tree}] building it by some other facts as we
    proceed
  \end{description}
  Even though it lacks flexibility and constraints us in many ways, we
  will explore a use case by implementing a very simple game before moving forward. 
\end{frame}
\begin{frame}
  \frametitle{8-Puzzle}
  The \href{http://en.wikipedia.org/wiki/15_puzzle}{8-puzzle} game is
  one of the earliest games exploited in AI. Numbered tiles, from 1 to
  8, are placed on a $3\times3$ grid.\par Starting from an initial grid
  configuration the objective is to reach a final state in which all
  tiles are ordered numerically with the last one left empty.\par
  The allowed moves consist in exchange the position of the the empty
  tile with one of its (at most) four neighbors.
  \begin{table}
    \setlength\tabcolsep{2pt}
    \centering
    \tiny
    \begin{tabular}{c c c c }
      \includegraphics[width=70pt]{Figures/8-puzzle-4}
      & \includegraphics[width=70pt]{Figures/8-puzzle-3}
      & \includegraphics[width=70pt]{Figures/8-puzzle-2}
      & \includegraphics[width=70pt]{Figures/8-puzzle-1}\\
    \end{tabular}
  \end{table}
  At this point of the course you should be pretty familiar with this game.
\end{frame}

\begin{frame}[fragile]
  \frametitle{8-Puzzle: Facts and Rules}
  Concerning states we can represent a grid configuration with a
  single fact. The information to be stored concerns \emph{the numbered
  positions} and \emph{the actual value} in them. For instance, we can use a special number
  like $-1$ to represent the empty tile value.\par\bigskip

  Even if inefficiently we can take advantage of the WM
  by simply asserting and never retracting (remind assertion and
  refraction properties).\par\bigskip

  We can have each rule modeling a single possible empty cell movement
  (24 rules total), something like:\par\bigskip

  \textbf{If} the grid has $X_1$ in the 1st position, -1 in the 2nd, $X_3$ in the
  third, $X_4$ in the fourth, $X_5$ in the fifth, \dots, \textbf{then} the new
  grid will be:  $X_1$ in the 1st position, $X_5$ in the 2nd, $X_3$ in the
  third, $X_4$ in the fourth, -1 in the fifth,\dots\par\bigskip

  We could reduce the number of rules by exploiting functional abstraction.\par

  
  
  
\end{frame}

\begin{frame}[fragile]
  \frametitle{8-Puzzle: Constraints and Strategy}
  Avoiding loops is done by the WM use we are doing and by enforcing
  to match only rules leading to unseen configurations.\par
  Refraction helps, but we need to avoid even to trigger rules that
  would assert and already asserted state.\par

  One possible solution is to put a constraint as a pattern in the LHSs:\par\bigskip

  \textbf{If} the grid has $X_1$ in the 1st position, -1 in the 2nd, $X_3$ in the
  third, $X_4$ in the fourth, $X_5$ in the fifth, \dots, \textbf{AND}
  there is \emph{not} $X_1$ in the 1st position, $X_5$ in the 2nd, $X_3$ in the
  third, $X_4$ in the fourth, -1 in the fifth\dots, \textbf{then} the new
  grid will be:  $X_1$ in the 1st position, $X_5$ in the 2nd, $X_3$ in the
  third, $X_4$ in the fourth, -1 in the fifth,\dots\par\bigskip

  The search strategy can be demanded to CLIPS by enforcing all rules
  to have the same salience and by employing the \textsf{breadth} or
  \textsf{depth} conflict resolution strategy.
  \begin{clips-code}[numers=none]
    (set-strategy depth)
  \end{clips-code}
  What is, for this particular problem, the best strategy to employ? why?\par

\end{frame}

\begin{frame}
  \frametitle{8-Puzzle: Constraints and Search Tree}
  As a last constraint we need rules to check for a state to be the
  final one (solution). This is simply done like in the previous
  fashion.\par\bigskip
  
  To build a search tree we can store its edge information as facts
  containing information about the two configurations of the
  world, the parent and child grids.\par\bigskip

  After a solution has been found we have to crawl back from that
  state to the root. To do this we need a simple rule that searches
  for a state matching for the parent attribute in our edge facts\par\bigskip

  

  
\end{frame}

\begin{frame}[fragile]
  \frametitle{Exercise}
  Implement such a problem solver for the 8-puzzle.\par\bigskip
  
  \textbf{Hint 1}:\par
  While programming apply a debugging strategy, inspect the WM after
  each activated rule execution. Moreover, keep track of strategy
  resolution by printing to stout the state that activates a rule and
  the one that the rule asserts.\par\bigskip
  
  \textbf{Hint 2}:\par
  To set the strategy and maybe constants use another file to by
  loaded with the \textsf{batch} function.\par\bigskip

  \textbf{Hint 3}:\par
  For the tree representation you can use something like this for
  edges, containing simply the fact addresses (pointers):
  \begin{clips-code}[numbers=none]
    (deftemplate move
        (slot parent (type FACT-ADDRESS) (default ?NONE))
        (slot next (type FACT-ADDRESS) (default ?NONE)))
  \end{clips-code}
      
\end{frame}

\begin{frame}
  \frametitle{Pros\&Cons}
  What \textbf{\emph{we gain for free}} is CLIPS inference routines and constructs,
  i.e. we do not have to implement a breadth first search or the
  neighbor space generation by ourselves (we just describe how they are generated
  via transition rules).\par\bigskip
  
  However, \emph{we cannot explore the search space in a more sophisticated way}.\par
  
  If we use CLIPS conflict resolution strategy for free, we cannot
  assign different priorities to rules. Neither it is possible to
  implement other search techniques like best-first search, $A^{*}$,
  and so on. We could implement the Manatthan distance heuristic
  score as a function, but we would have no way to embed its
  evaluation on the pairs of states.\par\bigskip

  A higher level representation for transition rules and states is
  needed.
\end{frame}

\begin{frame}
  \frametitle{Cannibals and Missionaries}
  
\end{frame}

\end{document}


%%% Local Variables:
%%% mode: latex
%%% TeX-engine: xetex
%%% TeX-master: t
%%% End:
